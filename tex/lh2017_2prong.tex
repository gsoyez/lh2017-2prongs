\documentclass[11pt,letterpaper]{article}
\pdfoutput=1
\usepackage{jheppub}

\usepackage[utf8]{inputenc}

\usepackage{color}
\usepackage{graphicx}
\usepackage{tabularx}
\usepackage{xspace}

\usepackage{verbatim}
\usepackage{amsmath}
\usepackage{amssymb}
\usepackage[caption=false]{subfig}
\usepackage{url}
\usepackage{bbold}
\usepackage{slashed}
\usepackage{array}

\usepackage{multirow}
\usepackage{threeparttable}
\usepackage{paralist}

\newcommand{\GeV}{\text{GeV}}
\newcommand{\TeV}{\text{TeV}}
\newcommand{\SO}{\text{SO}}
\newcommand{\SU}{\text{SU}}
\newcommand{\SM}{\text{SM}}

\newcommand{\U}{\text{U}}
\newcommand{\CKM}{\text{CKM}}
\newcommand{\eff}{\text{eff}}

\newcommand{\genang}[2]{{\lambda^{#1}_{#2}}}


\newcommand{\ev}{\text{event}}
\newcommand{\jet}{\text{jet}}
\newcommand{\jets}{\text{jets}}
\newcommand{\subj}{\text{subjet}}
\newcommand{\subjs}{\text{subjets}}
\newcommand{\cut}{\text{cut}}
\newcommand{\trim}{\text{trim}}
\newcommand{\Ecut}{E_{{\rm cut}}}

\newcommand{\ptc}{p_{T{\rm cut}}}
\newcommand{\ptsubc}{p_{T{\rm subcut}}}

\newcommand{\sub}{\text{sub}}
\newcommand{\miss}{\text{miss}}

\newcommand{\pythia}{\textsc{Pythia~8}\xspace}
\newcommand{\herwig}{\textsc{Herwig++}\xspace}
\newcommand{\eventtwo}{\textsc{Event2}\xspace}
\newcommand{\vincia}{\textsc{Vincia}\xspace}
\newcommand{\sherpa}{\textsc{Sherpa}\xspace}

\newcommand{\FastJet}{\textsc{FastJet}\xspace}
\newcommand{\MadGraph}{\textsc{MadGraph}\xspace}

\newcommand{\df}{\text{d}}
\newcommand{\vev}[1]{\langle #1 \rangle}


\DeclareRobustCommand{\Sec}[1]{Sec.~\ref{#1}}
\DeclareRobustCommand{\Secs}[2]{Secs.~\ref{#1} and \ref{#2}}
\DeclareRobustCommand{\Secss}[3]{Secs.~\ref{#1}, \ref{#2}, and \ref{#3}}
\DeclareRobustCommand{\App}[1]{App.~\ref{#1}}
\DeclareRobustCommand{\Tab}[1]{Table~\ref{#1}}
\DeclareRobustCommand{\Tabs}[2]{Tables~\ref{#1} and \ref{#2}}
\DeclareRobustCommand{\Fig}[1]{Fig.~\ref{#1}}
\DeclareRobustCommand{\Figs}[2]{Figs.~\ref{#1} and \ref{#2}}
\DeclareRobustCommand{\Figss}[3]{Figs.~\ref{#1}, \ref{#2}, and \ref{#3}}
\DeclareRobustCommand{\Eq}[1]{Eq.~(\ref{#1})}
\DeclareRobustCommand{\Eqs}[2]{Eqs.~(\ref{#1}) and (\ref{#2})}
\DeclareRobustCommand{\Eqss}[3]{Eqs.~(\ref{#1}), (\ref{#2}), and (\ref{#3})}
\DeclareRobustCommand{\Ref}[1]{Ref.~\cite{#1}}
\DeclareRobustCommand{\Refs}[1]{Refs.~\cite{#1}}

\newcommand{\be}{\begin{equation}}
\newcommand{\ee}{\end{equation}}
\newcommand{\nn}{\nonumber}

\renewcommand{\textfraction}{0.10}
\renewcommand{\topfraction}{0.90}
\renewcommand{\bottomfraction}{0.90}
\renewcommand{\floatpagefraction}{0.65}

%% Reference commands %%
\newcommand{\mb}[1]{\boldsymbol{#1}}
\newcommand{\bm}[1]{\boldsymbol{#1}}
\newcommand{\mbo}[1]{\boldsymbol{\overline{#1}}}

\usepackage{xspace}


\def\Tr{\mathop{\rm Tr}}
\newcommand{\rep}[1]{\mathbf{#1}}
\newcommand{\conjrep}[1]{\overline{\mathbf{#1}}}


\renewcommand{\a}{\alpha}
\renewcommand{\b}{\beta}
\newcommand{\e}{\epsilon}
\newcommand{\D}{\Delta}
\renewcommand{\l}{\lambda}
\renewcommand{\th}{\theta}
\newcommand{\bq}{\bar{q}}
\newcommand{\zcut}{z_{\rm cut}}

\newcommand{\IZ}{\mathbb{Z}}
\newcommand{\cD}{\mathcal{D}}
\newcommand{\cL}{\mathcal{L}}
\newcommand{\cR}{\mathcal{R}}
\newcommand{\cF}{\mathcal{F}}
\newcommand{\cI}{\mathcal{I}}
\newcommand{\cK}{\mathcal{K}}
\newcommand{\beq}{\begin{eqnarray}}
\newcommand{\eeq}{\end{eqnarray}}

\newcommand{\F}{\mathcal{F}}
\newcommand{\Ft}{\widetilde{\mathcal{F}}}
\newcommand{\G}{\mathcal{G}}
\newcommand{\Gt}{\widetilde{\mathcal{G}}}
\newcommand{\HH}{\mathcal{H}}
\newcommand{\HHt}{\widetilde{\mathcal{H}}}
\newcommand{\ord}[1]{\mathcal{O}\!\left(#1\right)}

\newcommand*\numcircledmod[1]{#1 \!\!\! \bigcirc}

\newcommand{\Njet}{\widetilde{N}_{\rm jet}}
\newcommand{\dN}[1]{\Delta_{#1}}
\newcommand{\dNpm}{\Delta_{2\pm}}
\newcommand{\dNp}{\Delta_{2+}}
\newcommand{\dNm}{\Delta_{2-}}
\newcommand{\dNtm}{\Delta_{3-}}

\newcommand{\cT}{\mathcal{T}}
\newcommand{\as}{\alpha_s}
\renewcommand{\angle}{\theta}

%\definecolor{darkgreen}{rgb}{0,0.5,0}
%\newcommand{\jdt}[1]{\textbf{\textcolor{darkgreen}{(#1 --jdt)}}}

%\definecolor{darkblue}{rgb}{0,0,0.5}
%\newcommand{\gs}[1]{\textbf{\textcolor{darkblue}{(#1 --gs)}}}


\begin{document}


\title{2-prong study}

\author[a]{Disha Bhatia,}
\author[a]{Reina Camacho,}
\author[a]{Grigorios Chachamis?,}
\author[a]{Suman Chatterjee,}
\author[a]{Frederic Dreyer,}
\author[a]{Deepak Kar,}
\author[a]{Peter Loch,}
\author[a]{Ian Moult,}
\author[a]{Ben Nachman?,}
\author[a]{Andreas Papaefstathiou,}
\author[a]{Tousik Samui,}
\author[a]{Andrzej Siodmok,}
\author[a]{Gregory Soyez,}
\emailAdd{gregory.soyez@cea.fr}
\author[a]{and Jesse Thaler}
\emailAdd{jthaler@mit.edu}

\affiliation[a]{Les Houches}

\abstract{YADA}

\maketitle

\section{Introduction}
\label{sec:introduction}

Variety of 2-prong taggers, want to understand behavior.  Focus on W bosons.  No b-tagging.  

Goals:  understand ATLAS/CMS differences, very high pT behavior,  interplay of jet radius, jet grooming, jet discrimination.  Groomed/ungroomed/dichroic observables.  Detector effects (simplified)

Also interesting in tagging longitudinal vs. transverse W bosons

\section{Preliminaries}

\subsection{Event Samples}

WW Pythia 8 pTW > 500 GeV

Need polarized Ws

Need dijet samples

Pileup Samples from pileup study

\subsection{Observables}

CMS:  mMDT mass, ungroomed N2
ATLAS:  trimmed mass, trimmed D2

\subsection{Detector Simulation}

Do we do DELPHES?

TowerGrid from Peter, included ATLAS/CMS B-field

Calorimetry vs. Particle Flow

\section{Particle-Level Tagging}

\subsection{Unpolarized Case}

\subsection{Longitudinal versus Transverse Polarization}

\subsection{Adjusting the Jet Radius}

\section{Impact of Detector Effects}

\subsection{Mass Resolution}

\subsection{Tagging Performance}

\subsection{Behavior in the Ultra-boosted Regime}

\section{Comparison of ATLAS and CMS Two-prong Strategies}

\section{Conclusions}

\begin{acknowledgments}

The work of GS is supported in part by the Paris-Saclay IDEX under the
IDEOPTIMALJE grant, by the French Agence Nationale de la Recherche,
under grant ANR-15-CE31-0016, and by the ERC Advanced Grant Higgs@LHC
(No.\ 321133).
%
The work of JT is supported by the DOE under grant contract numbers DE-SC-00012567 and DE-SC-00015476.

\end{acknowledgments}

\bibliographystyle{jhep}
\bibliography{lh2017_2prong}

\end{document}
