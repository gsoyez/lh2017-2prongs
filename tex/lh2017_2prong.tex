\documentclass[11pt,letterpaper]{article}
\pdfoutput=1
\usepackage{jheppub}

\usepackage[utf8]{inputenc}

\usepackage{color}
\usepackage{graphicx}
\usepackage{tabularx}
\usepackage{xspace}

\usepackage{verbatim}
\usepackage{amsmath}
\usepackage{amssymb}
\usepackage[caption=false]{subfig}
\usepackage{url}
\usepackage{bbold}
\usepackage{slashed}
\usepackage{array}

\usepackage{multirow}
\usepackage{threeparttable}
\usepackage{paralist}

\newcommand{\GeV}{\text{GeV}}
\newcommand{\TeV}{\text{TeV}}
\newcommand{\SO}{\text{SO}}
\newcommand{\SU}{\text{SU}}
\newcommand{\SM}{\text{SM}}

\newcommand{\U}{\text{U}}
\newcommand{\CKM}{\text{CKM}}
\newcommand{\eff}{\text{eff}}

\newcommand{\genang}[2]{{\lambda^{#1}_{#2}}}


\newcommand{\ev}{\text{event}}
\newcommand{\jet}{\text{jet}}
\newcommand{\jets}{\text{jets}}
\newcommand{\subj}{\text{subjet}}
\newcommand{\subjs}{\text{subjets}}
\newcommand{\cut}{\text{cut}}
\newcommand{\trim}{\text{trim}}
\newcommand{\Ecut}{E_{{\rm cut}}}

\newcommand{\ptc}{p_{T{\rm cut}}}
\newcommand{\ptsubc}{p_{T{\rm subcut}}}

\newcommand{\sub}{\text{sub}}
\newcommand{\miss}{\text{miss}}

\newcommand{\pythia}{\textsc{Pythia~8}\xspace}
\newcommand{\herwig}{\textsc{Herwig++}\xspace}
\newcommand{\eventtwo}{\textsc{Event2}\xspace}
\newcommand{\vincia}{\textsc{Vincia}\xspace}
\newcommand{\sherpa}{\textsc{Sherpa}\xspace}

\newcommand{\FastJet}{\textsc{FastJet}\xspace}
\newcommand{\MadGraph}{\textsc{MadGraph}\xspace}

\newcommand{\df}{\text{d}}
\newcommand{\vev}[1]{\langle #1 \rangle}


\DeclareRobustCommand{\Sec}[1]{Sec.~\ref{#1}}
\DeclareRobustCommand{\Secs}[2]{Secs.~\ref{#1} and \ref{#2}}
\DeclareRobustCommand{\Secss}[3]{Secs.~\ref{#1}, \ref{#2}, and \ref{#3}}
\DeclareRobustCommand{\App}[1]{App.~\ref{#1}}
\DeclareRobustCommand{\Tab}[1]{Table~\ref{#1}}
\DeclareRobustCommand{\Tabs}[2]{Tables~\ref{#1} and \ref{#2}}
\DeclareRobustCommand{\Fig}[1]{Fig.~\ref{#1}}
\DeclareRobustCommand{\Figs}[2]{Figs.~\ref{#1} and \ref{#2}}
\DeclareRobustCommand{\Figss}[3]{Figs.~\ref{#1}, \ref{#2}, and \ref{#3}}
\DeclareRobustCommand{\Eq}[1]{Eq.~(\ref{#1})}
\DeclareRobustCommand{\Eqs}[2]{Eqs.~(\ref{#1}) and (\ref{#2})}
\DeclareRobustCommand{\Eqss}[3]{Eqs.~(\ref{#1}), (\ref{#2}), and (\ref{#3})}
\DeclareRobustCommand{\Ref}[1]{Ref.~\cite{#1}}
\DeclareRobustCommand{\Refs}[1]{Refs.~\cite{#1}}

\newcommand{\be}{\begin{equation}}
\newcommand{\ee}{\end{equation}}
\newcommand{\nn}{\nonumber}

\renewcommand{\textfraction}{0.10}
\renewcommand{\topfraction}{0.90}
\renewcommand{\bottomfraction}{0.90}
\renewcommand{\floatpagefraction}{0.65}

%% Reference commands %%
\newcommand{\mb}[1]{\boldsymbol{#1}}
\newcommand{\bm}[1]{\boldsymbol{#1}}
\newcommand{\mbo}[1]{\boldsymbol{\overline{#1}}}

\usepackage{xspace}


\def\Tr{\mathop{\rm Tr}}
\newcommand{\rep}[1]{\mathbf{#1}}
\newcommand{\conjrep}[1]{\overline{\mathbf{#1}}}


\renewcommand{\a}{\alpha}
\renewcommand{\b}{\beta}
\newcommand{\e}{\epsilon}
\newcommand{\D}{\Delta}
\renewcommand{\l}{\lambda}
\renewcommand{\th}{\theta}
\newcommand{\bq}{\bar{q}}
\newcommand{\zcut}{z_{\rm cut}}

\newcommand{\IZ}{\mathbb{Z}}
\newcommand{\cD}{\mathcal{D}}
\newcommand{\cL}{\mathcal{L}}
\newcommand{\cR}{\mathcal{R}}
\newcommand{\cF}{\mathcal{F}}
\newcommand{\cI}{\mathcal{I}}
\newcommand{\cK}{\mathcal{K}}
\newcommand{\beq}{\begin{eqnarray}}
\newcommand{\eeq}{\end{eqnarray}}

\newcommand{\F}{\mathcal{F}}
\newcommand{\Ft}{\widetilde{\mathcal{F}}}
\newcommand{\G}{\mathcal{G}}
\newcommand{\Gt}{\widetilde{\mathcal{G}}}
\newcommand{\HH}{\mathcal{H}}
\newcommand{\HHt}{\widetilde{\mathcal{H}}}
\newcommand{\ord}[1]{\mathcal{O}\!\left(#1\right)}

\newcommand*\numcircledmod[1]{#1 \!\!\! \bigcirc}

\newcommand{\Njet}{\widetilde{N}_{\rm jet}}
\newcommand{\dN}[1]{\Delta_{#1}}
\newcommand{\dNpm}{\Delta_{2\pm}}
\newcommand{\dNp}{\Delta_{2+}}
\newcommand{\dNm}{\Delta_{2-}}
\newcommand{\dNtm}{\Delta_{3-}}

\newcommand{\cT}{\mathcal{T}}
\newcommand{\as}{\alpha_s}
\renewcommand{\angle}{\theta}

%\definecolor{darkgreen}{rgb}{0,0.5,0}
%\newcommand{\jdt}[1]{\textbf{\textcolor{darkgreen}{(#1 --jdt)}}}

%\definecolor{darkblue}{rgb}{0,0,0.5}
%\newcommand{\gs}[1]{\textbf{\textcolor{darkblue}{(#1 --gs)}}}


\begin{document}


\title{2-prong study}

\author[a]{Disha Bhatia,}
\author[a]{Reina Camacho,}
\author[a]{Grigorios Chachamis?,}
\author[a]{Suman Chatterjee,}
\author[a]{Frederic Dreyer,}
\author[a]{Deepak Kar,}
\author[a]{Peter Loch,}
\author[a]{Ian Moult,}
\author[a]{Ben Nachman?,}
\author[a]{Andreas Papaefstathiou,}
\author[a]{Tousik Samui,}
\author[a]{Andrzej Siodmok,}
\author[a]{Gregory Soyez,}
\emailAdd{gregory.soyez@cea.fr}
\author[a]{and Jesse Thaler}
\emailAdd{jthaler@mit.edu}

\affiliation[a]{Les Houches}

\abstract{YADA}

\maketitle

\section{Introduction}
\label{sec:introduction}


\section{}

\begin{verbatim}

Two-Prong Substructure Subsubgroup

Notes from Wednesday, June 7, 10am Meeting

(Jesse taking notes, apologies for mistakes)

Boosted W Physics:
	Tag longitudinal vs. transverse W bosons
	How far in pT can you actually tag (detector effect)?
	Understanding ATLAS/CMS technology differences (need mimicking of detector)

Decisions on Benchmark:
	sqrt(s) = 14 TeV, <mu> = 50/200?
	anti-kT R = 0.8?  R = 1.0?
	Use FastJet for everything
	Establish GitHub repo for code
	
Signal Samples:
	WW Pythia 8 pTW > 500 GeV (want 1600 GeV as well, narrow decays to push calorimeter granularity)
	(Need polarized Ws!)

Background Samples:
	Dijet samples available (from pileup study)

Pileup Samples:
	10^7 pileup (from pileup study)

Detector Issues:
	Peter has "stupid" detector simulation (smearing model:  TowerGrid)
	ECAL grid 0.025 x 0.025
	HCAL grid 0.1 x 0.1
	-> PseudoJets
	Acceptance cuts on towers
	Fluctuations on towers
	Eta/phi fixed to center of tower, treated as massless
	(Comparison to DELPHES?)

ATLAS vs. CMS (vs. truth):
	ATLAS B-field is half of CMS B-field
	Particle flow versus not particle flow
	CMS corrects direction of charged particles
	ATLAS does not correct for this magnetic field effect
	Difference of acceptance as well
	Another issue is resolution at low pT (versus acceptance)

Why different?:
	Independent of resolution:  Grooming the observables versus grooming the mass
	ATLAS (D2 + trimming) vs. CMS (N2 + soft drop)
	Different in pileup subtraction method (complicated in ATLAS because of out-of-time pileup)

Metrics:
	How do we assess performance?
	W mass peak, ...
	
Next Steps:
	Need polarized W sample (Reina):  X -> W_L W_L,  Y -> W_T W_T
	Need to add UE (Gregory) to dijet samples
	Prepare Herwig samples (Andrzej)
	Madgraph?
	Samples:  Polarized W, Unpolarized W, Dijet (no need for W + jet?)
	Scales:  pT > 500 GeV, 1000 GeV, 2000 GeV
	Default: <mu> = 50
	Force hadronic W decay
	Look at distributions and ROC curves
	Do Madgraph decay?	

File format:
	#event <id>
	px py pz m pdgId
	end

Reina/Deepak/Disha:  event generation
Gregory/Frederic:  software
Suman/Tousik:  plotting
Andrzej/Peter/Andreas:  Herwig
Peter:  TowerGrid simulation
Jesse:  moral support/documentation
Ian:  observable catalog

\end{verbatim}

\begin{acknowledgments}

The work of GS is supported in part by the Paris-Saclay IDEX under the
IDEOPTIMALJE grant, by the French Agence Nationale de la Recherche,
under grant ANR-15-CE31-0016, and by the ERC Advanced Grant Higgs@LHC
(No.\ 321133).
%
The work of JT is supported by the DOE under grant contract numbers DE-SC-00012567 and DE-SC-00015476.

\end{acknowledgments}

\bibliographystyle{jhep}
\bibliography{lh2017_2prong}

\end{document}
